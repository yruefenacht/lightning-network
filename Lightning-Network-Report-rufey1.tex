\documentclass[a4paper, 12pt]{report}

% Charset
\usepackage[utf8]{inputenc}
% Language
\usepackage[ngerman]{babel}
% Font
\usepackage[default]{sourcesanspro}
\usepackage[T1]{fontenc}
\usepackage{microtype}
% Graphics
\usepackage{graphicx}
\usepackage{fancyhdr}
% Indexing
\usepackage{index}
\makeindex

\usepackage{blindtext}

\begin{document}
\title{\Large{\textbf{Helloo bröther}}}
\author{Yannick Rüfenacht}
\date{21.10.2020}

\maketitle
\tableofcontents

\chapter{Bitcoin}
\section{Das Problem mit Bitcoin}
\blindtext[5]

\end{document}
